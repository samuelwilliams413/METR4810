\documentclass[letterpaper, 12 pt]{article}

\usepackage{url}
\usepackage{color}
\usepackage{hyperref}
\hypersetup{
    colorlinks,
    linktoc=all,
    citecolor=black,
    filecolor=black,
    linkcolor=black,
    urlcolor=black
}

% -------------------------------------------------------------------------------------
% BEGIN DOCUMENT
% -------------------------------------------------------------------------------------
\begin{document}
\title{Bad Bitch User Manual}
\author{Team 10}
\maketitle
\pagestyle{empty}


%%%Warning
\newpage
\vspace*{13cm}
\section*{Warning!}
The Bad Bitch was designed and build only to recover the submarine. The user assumes the responsibility of using it for any other purpose!

% -------------------------------------------------------------------------------------
% TABLE OF CONTENTS
% -------------------------------------------------------------------------------------
\newpage
\tableofcontents
\newpage

\section{Introduction}
The Bad Bitch is used to recover the sunken miniature submarine from the tank. This user manual describes how to assemble the Bad Bitch and how to successfully use it to accomplish the recovery mission. The manual contains also hints about the required software.
\section{System description}
The Bad Bitch consists of a buoy floating on the sea surface and a rover equipped with a camera and a claw. The Bad Bitch is controlled through a computer keyboard. Table \ref{tab:components} gives the complete list of parts that compose the Bad Bitch.
\begin{table}[h]
\begin{center}
\caption{List of parts composing the Bad Bitch.}
\label{tab:components}
\vspace{0.5cm}
\begin{tabular}{|c|c|}
\hline
\textbf{Part} & \textbf{Quantity}\\
\hline
Claw with arm & 1\\
\hline
Rover body & 1\\
\hline
Rover wheels & 4\\
\hline
DB cable & 1\\
\hline
Camera receiver & 1\\

\hline


\end{tabular}
\end{center}
\end{table}
\section{Assembly}
Describe how to assemble the thing

Required tools
\begin{itemize}
\item Screw drive
\item ..
\end{itemize}

Put part A on B..
\section{Required software}
\subsection{Bluetooth communication}
The Bad Bitch is remotely controlled through bluetooth serial communication. After establishing the bluetooth connection with the Bad Bitch, you have several ways to communicate with the system. One possibility is to use the open source software PUTTY. PUTTY is an SSH and telnet client. To download and install PUTTY, use the following link to select your platform and follow the instructions:
\url{https://www.chiark.greenend.org.uk/~sgtatham/putty/latest.html}.
\subsection{Camera software}

\section{Starting the Bad Bitch}
To start the Bad Bitch, press the Start Button on the buoy.
\subsection{Starting the camera}
The camera receiver has to be connected to one of the computer USB ports. Then start  
\subsection{Establishing the bluetooth connection}
Establishing the Bluetooth models depends on the used platform, please refer to your platform bluetooth documentation. Here is a description for Windows users. Once the Bad Bitch is powered, in the computer under the Bluetooth properties, select "Add a device". The default device name of the Bluetooth module is "Bad Bitch". Double click "Bad Bitch" and select the "Hardware" tab, you can see which COM port the module is connected.
\subsection{Using serial communication}
As stated above, you may use any terminal programs for the serial communication. Here are the instruction for using PUTTY.
\begin{itemize}
\item Select the ‘Serial’ radio button.
\item Type the COM port number that the Bluetooth device is connecting to.
\item Set the default baud rate to 9600.
\item Click "Open". A terminal should be opened. The first line should contain the "Ready!" message which indicates that the Bad Bit is ready for receiving commands.
\end{itemize}
\section{System control}
The Bad Bitch is controlled through the keyboard. Table \ref{tab:keys} summarizes the keys that control the Bad Bitch. After pressing a key, it is not required to press the enter key. Once the command has been executed, a feedback message is displayed in the terminal followed by the current status of all actuators.
\begin{table}[h]
\begin{center}
\caption{Keys controlling the system.}
\label{tab:keys}
\vspace{0.5cm}
\begin{tabular}{|c|c|}
\hline
\textbf{Character} & \textbf{Action}\\
\hline
'q' & Move the winch rope up\\
\hline
'a' & Move the winch rope down\\
\hline
'x' & Hold the winch\\
\hline
'z' & Release the winch\\
\hline 
'w' & Move the claw servo 5$^{\circ}$ toward closing\\
\hline  
's' & Move the claw servo 5$^{\circ}$ toward opening\\
\hline
'e' & Move the camera servo 5$^{\circ}$ to the left\\
\hline  
'r' & Move the camera servo 5$^{\circ}$ to the right\\
\hline
'i' & Move the rover forward\\
\hline
'm' & Move the rover backward\\
\hline
'k' or space & Stop the rover\\
\hline
'l' & Turn the rover to the right\\
\hline
'j' & Turn the rover to the left\\
\hline

\end{tabular}
\end{center}
\end{table}
% -------------------------------------------------------------------------------------
% REFERENCES
% -------------------------------------------------------------------------------------
%\bibliography{}

% -------------------------------------------------------------------------------------
% END DOCUMENT
% -------------------------------------------------------------------------------------
\end{document}